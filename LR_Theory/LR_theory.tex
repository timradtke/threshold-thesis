\documentclass[12pt,]{article}
\usepackage{lmodern}
\usepackage{amssymb,amsmath}
\usepackage{ifxetex,ifluatex}
\usepackage{fixltx2e} % provides \textsubscript
\ifnum 0\ifxetex 1\fi\ifluatex 1\fi=0 % if pdftex
  \usepackage[T1]{fontenc}
  \usepackage[utf8]{inputenc}
\else % if luatex or xelatex
  \ifxetex
    \usepackage{mathspec}
  \else
    \usepackage{fontspec}
  \fi
  \defaultfontfeatures{Ligatures=TeX,Scale=MatchLowercase}
\fi
% use upquote if available, for straight quotes in verbatim environments
\IfFileExists{upquote.sty}{\usepackage{upquote}}{}
% use microtype if available
\IfFileExists{microtype.sty}{%
\usepackage{microtype}
\UseMicrotypeSet[protrusion]{basicmath} % disable protrusion for tt fonts
}{}
\usepackage[margin=1in]{geometry}
\usepackage{hyperref}
\PassOptionsToPackage{usenames,dvipsnames}{color} % color is loaded by hyperref
\hypersetup{unicode=true,
            pdftitle={LR Theory},
            pdfauthor={Tim Radtke},
            colorlinks=true,
            linkcolor=Maroon,
            citecolor=Blue,
            urlcolor=blue,
            breaklinks=true}
\urlstyle{same}  % don't use monospace font for urls
\usepackage{graphicx,grffile}
\makeatletter
\def\maxwidth{\ifdim\Gin@nat@width>\linewidth\linewidth\else\Gin@nat@width\fi}
\def\maxheight{\ifdim\Gin@nat@height>\textheight\textheight\else\Gin@nat@height\fi}
\makeatother
% Scale images if necessary, so that they will not overflow the page
% margins by default, and it is still possible to overwrite the defaults
% using explicit options in \includegraphics[width, height, ...]{}
\setkeys{Gin}{width=\maxwidth,height=\maxheight,keepaspectratio}
\IfFileExists{parskip.sty}{%
\usepackage{parskip}
}{% else
\setlength{\parindent}{0pt}
\setlength{\parskip}{6pt plus 2pt minus 1pt}
}
\setlength{\emergencystretch}{3em}  % prevent overfull lines
\providecommand{\tightlist}{%
  \setlength{\itemsep}{0pt}\setlength{\parskip}{0pt}}
\setcounter{secnumdepth}{5}
% Redefines (sub)paragraphs to behave more like sections
\ifx\paragraph\undefined\else
\let\oldparagraph\paragraph
\renewcommand{\paragraph}[1]{\oldparagraph{#1}\mbox{}}
\fi
\ifx\subparagraph\undefined\else
\let\oldsubparagraph\subparagraph
\renewcommand{\subparagraph}[1]{\oldsubparagraph{#1}\mbox{}}
\fi

%%% Use protect on footnotes to avoid problems with footnotes in titles
\let\rmarkdownfootnote\footnote%
\def\footnote{\protect\rmarkdownfootnote}

%%% Change title format to be more compact
\usepackage{titling}

% Create subtitle command for use in maketitle
\newcommand{\subtitle}[1]{
  \posttitle{
    \begin{center}\large#1\end{center}
    }
}

\setlength{\droptitle}{-2em}
  \title{LR Theory}
  \pretitle{\vspace{\droptitle}\centering\huge}
  \posttitle{\par}
  \author{Tim Radtke}
  \preauthor{\centering\large\emph}
  \postauthor{\par}
  \date{}
  \predate{}\postdate{}

\usepackage[retainorgcmds]{IEEEtrantools}
\usepackage{bm}
\usepackage{amsmath}
\usepackage{bbm}
\newtheorem{theorem}{Theorem}
\newtheorem{lemma}{Lemma}
\newcommand{\KL}{\,\text{KL}}
\newcommand{\der}{\,\text{d}}

\begin{document}
\maketitle

\section{Kullback-Leibler Divergence
Strategy}\label{kullback-leibler-divergence-strategy}

We start with a definition of the Likelihood Ratio. See for example
Example 6 in \url{http://math.arizona.edu/~jwatkins/s-lrt.pdf}.

In our strategy, we will have at time \(t\) a sample of observations for
arm \(i\) of size \(T_i(t)\): \(\mathbb{x} = x_1, ..., x_{T_i(t)}\).
Since we are only considering Bernoulli distributions at this point, the
Likelihood for a sample from distribution \(Ber(\mu)\) is given by

\[ 
L(\mu|\mathbb{x}) = (1-\mu)^{T_i(t) - \sum_{s = 1}^{T_i(t)} x_s} \mu^{\sum_{s = 1}^{T_i(t)} x_s}
\]

We want to test whether the sample \(\mathbb{x}\) with maximum
likelihood estimate (\(H_1: \mu = \hat{\mu}\)) is actually different
from the standard hypothesis that it is equal to the threshold \(\tau\)
(\(H_0: \mu = \tau\)). To test this, we can use the likelihood ratio:

\[
\Lambda = \frac{L(\tau|\mathbb{x})}{L(\hat{\mu}|\mathbb{x})} = \frac{\tau^{\hat{\mu}T_i(t)}(1-\tau)^{(1-\hat{\mu})T_i(t)} }{\hat{\mu}^{\hat{\mu}T_i(t)}(1-\hat{\mu})^{(1-\hat{\mu})T_i(t)}}
\] If the likelihood ratio is large, then it's quite likely that the
sample was generated by a Bernoulli distribution with parameter
\(\tau\). If the ratio is small, then the maximum likelihood estimate is
very different from the threshold. The statistic should be smaller than
1 given that we compare to the maximum likelihood estimate.

In an adaptive sampling strategy that is based on this ratio, it is
natural to pull the arm that currently maximizes the ratio (an arm with
distribution close to \(\tau\)). Even if the true parameter \(\mu\) is
close to \(\tau\), the ratio will increase over time as \(T_i(t)\)
increases. Thus given two distributions with the same maximum likelihood
estimate, the strategy will pull the arm that has a smaller \(T_i(t)\).

Thus our strategy could be to pull arm \(i^*\) at time \(t+1\) that
maximizes \(\tilde{B}_i(t) = \Lambda_i\). Of course,
\(\arg \max_i \Lambda_i(t) = \arg \min_i -\log(\Lambda_i(t))\). And so
an equivalent strategy is to pull
\(i^*(t) = \arg \min_i -\log(\Lambda_i(t))\). We now show that this is
equivalent to \(i^*(t) = \arg \min_i T_i(t)\hat{\KL}(\hat{\mu}||\tau)\).

We have

\begin{align*}
-\log(\Lambda_i(t)) & = \hat{\mu}T_i(t) \log(\frac{\hat{\mu}_i(t)}{\tau}) + (1-\hat{\mu})T_i(t) \log(\frac{1-\hat{\mu}_i(t)}{1-\tau}) \\
& = T_i(t) [\hat{\mu}_i(t) \log(\frac{\hat{\mu}_i(t)}{\tau}) + (1-\hat{\mu}_i(t)) \log(\frac{1-\hat{\mu}_i(t)}{1-\tau})] \\
& = T_i(t) \der(\hat{\mu}_i(t), \tau) \\
& = B_i(t)
\end{align*}

This notation of the index will prove useful in theoretical derivations,
while the likelihood ratio is advantageous in the implementation of the
algorithm, as it can be computed even when \(\hat{\mu}=0\) or
\(\hat{\mu}=1\), which will always occur in the beginning of the
sampling.

\subsection{Comparison with Other
Indices}\label{comparison-with-other-indices}

If we consider \(B_i^{APT}(t) = T_i(t) \der(\hat{\mu}_i(t), \tau)\), we
see a similarity in the index that first showed up in Locatelli et al.
with the APT strategy where
\(B^{APT}_i(t) = \sqrt{T_i(t)} \Delta_i(t) = \sqrt{T_i(t)} |\mu_i - \tau|\)
(if we consider the case where \(\epsilon = 0\)). This format for the
strategy's index was inherited by Zhong et al. (2017) for their EVT
algorithm which adapts both the term estimating the difference between
\(\hat{\mu}\) and \(\tau\), which I will now refer to as \(D_k(t)\), and
the confidence term of the current estimate. In Zhong et al., the index
is given by \(B_i^{EVT}(t) = D_i^{EVT}(t)S^{EVT}_i(t)\), where the
difference term \(D_i^{EVT}(t) = D_i^{APT}(t) = |\mu_i(t) - \tau|\), but
where the confidence term \(S_i^{EVT}(t)\) is no longer equal to
\(S_i^{APT}(t)=\sqrt{T_i(t)}\) but
\(S_i^{EVT}(t) = \Big(\frac{a}{T_i(t)} + \sqrt{\frac{a}{T_i(t)}}\hat{\sigma}_i(t)\Big)\),
and thus depends on the current estimate of the variance,
\(\hat{\sigma}_i(t)\). Continuing this format of indices, the strategy
proposed above has an index that can be written as

\[
B_i(t) = S^{KL}_i(t) D_i^{KL}(t) = T_i(t) \der(\hat{\mu}_i(t), \tau)
\]

If we write the Kullback-Leibler divergence as approximated by a Normal
distribution,
\(\der(\hat{\mu}, \tau) \approx \frac{(\hat{\mu}-\tau)^2}{\tau(1-\tau)}\)
or using Pinsker's inequality to bound it as
\(\der(\hat{\mu}, \tau) \geq 2|\hat{\mu}-\tau|^2\), we see that these
approximations lead to the index that is used by the APT strategy. Thus,
the motivation behind the likelihood ratio based strategy is to improve
upon APT for cases where the approximation fails. We see that in the
case of Bernoulli distributions, the two strategies are equivalent for
\(\tau = 0.5\).

\subsection{Upper Bound for KL
Strategy}\label{upper-bound-for-kl-strategy}

Given the general schema for the index in both the APT and the EVT
strategy, the proofs for their respective upper bound on the expected
loss share share certain steps. Both proofs build on favorable events
which hold with large probability for certain levels of \(T\), \(T_i\),
\(H\). In both proofs, the authors bound the number of pulls that a
helpful arm receives; that is, an arm that received at least a share of
the budget \(T\) that is proportional to his contribution to the overall
complexity \(H\), and which is played in the final round \(T\). Both
strategies need to show that on the favorable event, after \(T\) rounds,
every arm has been pulled a suffient amount of times for
\(\hat{\mu}_i(T)\) to be seperated sufficiently from \(\tau\) to make a
correct classification of arm \(i\). In the design of the favorable
event, both strategies can use the fact that the \(|\hat{\mu} - \tau|\)
term in the index get the following upper bound
\(|\hat{\Delta}_i(t) - \Delta_i| \leq |\hat{\mu}_i(t)-\mu_i|\) on the
\(D_i^{APT}(t) = D_i^{EVT}(t) = \hat{\Delta}_i(t)\) part of the index.
Given this bound, both proofs use a favorable event on which
\(\hat{\mu}\) is very close to \(\mu\) with large probability.

Given that the Kullback-Leibler divergence is not a metric, and more
specifically, the triangle inequality does not hold, the favorable event
from the other proofs will not suffice to bound \(D_i^{KL}(t)\).

Given the lower bound on \(T_k(t)\) for the helpful arm \(k\), and the
lower and upper bound on \(\hat{\Delta}_k(t)\) and \(\hat{\Delta}_i(t)\)
through the favorable event (compare equation (10) in A.2 of Locatelli
et al.), all that remains to show the sufficiently seperated
\(\hat{\mu}\) from \(\tau\) is to lower bound the number of times the
not helpful arms are pulled, \(T_i(t)\). Locatelli et al. can derive
this easily from the other bounds by comparing the two indices
\(B_k(t)\) and \(B_i(t)\). Since at time \(t\) the arm \(k\) has been
pulled, the APT strategy has \(B_k^{APT}(t) \leq B_i^{APT}(t)\), since
the algorithm pulls the \(\arg \min\). Similar holds for the EVT
algorithm. And using the Kullback-Leibler distance instead of the
Likelihood-ratio, the KL based strategy also minimizes \(B_i(t)\)
(instead of maximizing \(\tilde{B}_i(t)\).

Thus, for the KL-based strategy, consider again for the helpful arm
\(k\) pulled at time \(t\), and some other arm \(i\):

\[
B_k(t) \leq B_i(t) \Leftrightarrow T_k(t) \der(\hat{\mu}_k(t), \tau) \leq T_i(t) \der(\hat{\mu}_i(t), \tau)
\]

Starting from here, our goal will be to show that some kind of
seperation of the arms holds on the favorable event, where the favorable
event is set up such that the above quantities are bounded and the
seperation condition can be checked. On the way, the favorable event
will have an impact on how the complexity \(H\) will be defined for this
problem. In the end, we want to derive an upper bound of rate
\(\exp(-\frac{T}{H})\) on the error probability.


\end{document}
