\documentclass[12pt,]{article}
\usepackage{lmodern}
\usepackage{amssymb,amsmath}
\usepackage{ifxetex,ifluatex}
\usepackage{fixltx2e} % provides \textsubscript
\ifnum 0\ifxetex 1\fi\ifluatex 1\fi=0 % if pdftex
  \usepackage[T1]{fontenc}
  \usepackage[utf8]{inputenc}
\else % if luatex or xelatex
  \ifxetex
    \usepackage{mathspec}
  \else
    \usepackage{fontspec}
  \fi
  \defaultfontfeatures{Ligatures=TeX,Scale=MatchLowercase}
\fi
% use upquote if available, for straight quotes in verbatim environments
\IfFileExists{upquote.sty}{\usepackage{upquote}}{}
% use microtype if available
\IfFileExists{microtype.sty}{%
\usepackage{microtype}
\UseMicrotypeSet[protrusion]{basicmath} % disable protrusion for tt fonts
}{}
\usepackage[margin=1in]{geometry}
\usepackage{hyperref}
\PassOptionsToPackage{usenames,dvipsnames}{color} % color is loaded by hyperref
\hypersetup{unicode=true,
            pdftitle={The Thresholding Bandit Problem},
            pdfauthor={Tim Radtke},
            colorlinks=true,
            linkcolor=Maroon,
            citecolor=Blue,
            urlcolor=blue,
            breaklinks=true}
\urlstyle{same}  % don't use monospace font for urls
\usepackage{graphicx,grffile}
\makeatletter
\def\maxwidth{\ifdim\Gin@nat@width>\linewidth\linewidth\else\Gin@nat@width\fi}
\def\maxheight{\ifdim\Gin@nat@height>\textheight\textheight\else\Gin@nat@height\fi}
\makeatother
% Scale images if necessary, so that they will not overflow the page
% margins by default, and it is still possible to overwrite the defaults
% using explicit options in \includegraphics[width, height, ...]{}
\setkeys{Gin}{width=\maxwidth,height=\maxheight,keepaspectratio}
\IfFileExists{parskip.sty}{%
\usepackage{parskip}
}{% else
\setlength{\parindent}{0pt}
\setlength{\parskip}{6pt plus 2pt minus 1pt}
}
\setlength{\emergencystretch}{3em}  % prevent overfull lines
\providecommand{\tightlist}{%
  \setlength{\itemsep}{0pt}\setlength{\parskip}{0pt}}
\setcounter{secnumdepth}{5}
% Redefines (sub)paragraphs to behave more like sections
\ifx\paragraph\undefined\else
\let\oldparagraph\paragraph
\renewcommand{\paragraph}[1]{\oldparagraph{#1}\mbox{}}
\fi
\ifx\subparagraph\undefined\else
\let\oldsubparagraph\subparagraph
\renewcommand{\subparagraph}[1]{\oldsubparagraph{#1}\mbox{}}
\fi

%%% Use protect on footnotes to avoid problems with footnotes in titles
\let\rmarkdownfootnote\footnote%
\def\footnote{\protect\rmarkdownfootnote}

%%% Change title format to be more compact
\usepackage{titling}

% Create subtitle command for use in maketitle
\newcommand{\subtitle}[1]{
  \posttitle{
    \begin{center}\large#1\end{center}
    }
}

\setlength{\droptitle}{-2em}
  \title{The Thresholding Bandit Problem}
  \pretitle{\vspace{\droptitle}\centering\huge}
  \posttitle{\par}
  \author{Tim Radtke}
  \preauthor{\centering\large\emph}
  \postauthor{\par}
  \predate{\centering\large\emph}
  \postdate{\par}
  \date{4/23/2017}

\usepackage[retainorgcmds]{IEEEtrantools}
\usepackage{bm}
\usepackage{amsmath}
\usepackage{bbm}
\newtheorem{theorem}{Theorem}
\newcommand{\KL}{\,\text{KL}}
\newcommand{\der}{\,\text{d}}

\begin{document}
\maketitle

The \emph{Thresholding Bandit Problem} described in Locatelli et al.
(2016) can be considered a variant of a variant. It can be framed into
the wider literature of \emph{Pure Exploration} multi-armed bandit
problems. In particular, it shares characteristics with the
\emph{Top-\(m\)} problem. This problem is concerned with finding the
\(m\) best arms as described by the means of their corresponding
distributions. This in turn is similar to the \emph{combinatorial
bandit} problem, which also aims at finding the \(m\) best arms.
However, it is able to pull several arms at once: Think of an online
shop that shows five recommended products on a product detail page.
These five recommended products might each be represented by an arm, and
we look for the products with the largest mean conversion rate. The
thresholding bandit problem we discuss here, however, is concerned with
pulling a single arm at a time. And so a more appropriate situation is
that of a website presenting a banner. Again, think of an online shop
trying to promote a certain category. The content team came up with a
number of different designs for the banner, and it's not clear how many
click-throughs they will gather.

The idea now is that we would like to classify the banners into two
distinct groups: A group with a mean conversion rate \(\mu_i\) below
threshold \(\tau\), and a group with mean conversion rate \(\mu_i\)
above threshold \(\tau\). This might be the optimization problem when we
are concerned with not falling below a certain minimum level
click-through rate with the banners we're choosing.

If it turns out that in general we can find relatively quickly whether
arms are above or below a threshold, then this kind of test can be used
to safeguard against very bad versions in a test. Before we move to a
Top-\(m\) test, we might want to run a thresholding version and then
continue only with the arms classified into the group above the
threshold. This might be justified when adjusting parts of the checkout
process of an online shop, where the conversion rate or the average
order value should not drop below a threshold.

\subsection{Setup}\label{setup}

In any case, the problem boils down to the following. As standard in
multi-armed bandit problems, we have \(K\) arms
\(\mathcal{A} = \{1, \dots K\}\). We can pull arm \(k\) at time \(t\) to
collect feedback in form of the random variable \(X_{k,t}\) which is
distributed accoring to the arm's distribution \(\nu_k\). In general, we
are concerned with estimating for each arm \(i\) the mean \(\mu_i\) of
its distribution \(\nu_i\). In contrast to most bandit problems,
however, we do not directly compare arms with each other by comparing
their means. In the case of pure exploration bandits, it is for example
necessary to compare the arms because one aims to find the best one. In
the case of the thresholding bandit, however, we compare each arm's mean
\(\mu_i\) individually against the threshold \(\tau \in \mathbb{R}\)
which is known and fixed before the experiment. One might compare this
to a pure exploration bandit in which the mean of the best arm is known
upfront. We will see later that this fact leads to advantages in the
design of algorithms which are not as readily available in other bandit
settings.


\end{document}
