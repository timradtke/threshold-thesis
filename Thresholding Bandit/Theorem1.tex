\documentclass[12pt,]{article}
\usepackage{lmodern}
\usepackage{amssymb,amsmath}
\usepackage{ifxetex,ifluatex}
\usepackage{fixltx2e} % provides \textsubscript
\ifnum 0\ifxetex 1\fi\ifluatex 1\fi=0 % if pdftex
  \usepackage[T1]{fontenc}
  \usepackage[utf8]{inputenc}
\else % if luatex or xelatex
  \ifxetex
    \usepackage{mathspec}
  \else
    \usepackage{fontspec}
  \fi
  \defaultfontfeatures{Ligatures=TeX,Scale=MatchLowercase}
\fi
% use upquote if available, for straight quotes in verbatim environments
\IfFileExists{upquote.sty}{\usepackage{upquote}}{}
% use microtype if available
\IfFileExists{microtype.sty}{%
\usepackage{microtype}
\UseMicrotypeSet[protrusion]{basicmath} % disable protrusion for tt fonts
}{}
\usepackage[margin=1in]{geometry}
\usepackage{hyperref}
\PassOptionsToPackage{usenames,dvipsnames}{color} % color is loaded by hyperref
\hypersetup{unicode=true,
            pdftitle={Theorems and Proofs},
            pdfauthor={Tim Radtke},
            colorlinks=true,
            linkcolor=Maroon,
            citecolor=Blue,
            urlcolor=blue,
            breaklinks=true}
\urlstyle{same}  % don't use monospace font for urls
\usepackage{graphicx,grffile}
\makeatletter
\def\maxwidth{\ifdim\Gin@nat@width>\linewidth\linewidth\else\Gin@nat@width\fi}
\def\maxheight{\ifdim\Gin@nat@height>\textheight\textheight\else\Gin@nat@height\fi}
\makeatother
% Scale images if necessary, so that they will not overflow the page
% margins by default, and it is still possible to overwrite the defaults
% using explicit options in \includegraphics[width, height, ...]{}
\setkeys{Gin}{width=\maxwidth,height=\maxheight,keepaspectratio}
\IfFileExists{parskip.sty}{%
\usepackage{parskip}
}{% else
\setlength{\parindent}{0pt}
\setlength{\parskip}{6pt plus 2pt minus 1pt}
}
\setlength{\emergencystretch}{3em}  % prevent overfull lines
\providecommand{\tightlist}{%
  \setlength{\itemsep}{0pt}\setlength{\parskip}{0pt}}
\setcounter{secnumdepth}{5}
% Redefines (sub)paragraphs to behave more like sections
\ifx\paragraph\undefined\else
\let\oldparagraph\paragraph
\renewcommand{\paragraph}[1]{\oldparagraph{#1}\mbox{}}
\fi
\ifx\subparagraph\undefined\else
\let\oldsubparagraph\subparagraph
\renewcommand{\subparagraph}[1]{\oldsubparagraph{#1}\mbox{}}
\fi

%%% Use protect on footnotes to avoid problems with footnotes in titles
\let\rmarkdownfootnote\footnote%
\def\footnote{\protect\rmarkdownfootnote}

%%% Change title format to be more compact
\usepackage{titling}

% Create subtitle command for use in maketitle
\newcommand{\subtitle}[1]{
  \posttitle{
    \begin{center}\large#1\end{center}
    }
}

\setlength{\droptitle}{-2em}
  \title{Theorems and Proofs}
  \pretitle{\vspace{\droptitle}\centering\huge}
  \posttitle{\par}
  \author{Tim Radtke}
  \preauthor{\centering\large\emph}
  \postauthor{\par}
  \predate{\centering\large\emph}
  \postdate{\par}
  \date{4/23/2017}

\usepackage[retainorgcmds]{IEEEtrantools}
\usepackage{bm}
\usepackage{amsmath}
\usepackage{bbm}
\newtheorem{theorem}{Theorem}
\newtheorem{lemma}{Lemma}
\newcommand{\KL}{\,\text{KL}}
\newcommand{\der}{\,\text{d}}

\begin{document}
\maketitle

\subsection{Theorem 1}\label{theorem-1}

In Theorem 1 of Locatelli et al. (2016), a lower bound for the
thresholding bandit problem is shown. The basic idea is as follows. The
problem formulation considered in the paper describes bandit problems
for sub-Gaussian distributions. To show a lower bound, we show that
every algorithm is expected to make a certain error even if we restrict
ourselves to a very restricted scenario. Even then, any algorithm will
have a certain level of risk with which it cannot distinguish between
the actual setting and slightly changed settings (under which the former
answer leads to regret).

To be more specific, consider a case in which we set \(\tau = 0\) and
\(\epsilon = 0\) w.l.o.g. If we then choose all arms to have normally
distributed feedback with means \(\mu_i\), we see that for each arm
\(i\) the distance from the threshold is given by
\(\Delta_i = |\mu_i - \tau| + \epsilon = \mu_i\). Thus, assuming a
variance of 1 for every arm, we have for every arm the distribution
\(\nu_i = \mathcal{N}(\mu_i,1) = \mathcal{N}(\Delta_i,1)\). Furthermore,
define for each arm \(i \in \{1, \dots, K\}\) the alternative
distribution \(\nu_i' = \mathcal{N}(-\Delta_i,1)\). It is easy to see
that in this case, this is equivalent to a ``flip around the
threshold''. Arms that originally have a mean larger than the threshold
have a mean smaller than the threshold after the flip.

The idea now is that every arm \(i \in \{1,\dots,K\}\) generates a
sample for every time point \(t\in \{1, \dots, T\}\) from its
distribution \(\nu_i\) before the algorithm starts to choose arms. Thus,
we have a \(T \times K\) table of realized observations from the product
distribution given by
\(\mathcal{B}^0 = \nu_1^0 \otimes \ldots \otimes \nu_K^0 = \nu_1^0 \otimes \ldots \otimes \nu_K\).
This is a product distribution where the \(K\) different means all lie
above the threshold. We would hope that an algorithm classifies the
means accordingly. However, we will show in what follows that any
algorithm makes a certain error by not being able to distinguish
\(\mathcal{B}^0\) from at least one of \(K\) alternative product
distributions \(\mathcal{B}^i, i \in \{1, \dots, K\}\), of which each
differs only slightly in one arm from the original distribution, and has
the same overall complexity (as defined by
\(H = \sum_{i=1}^K (\Delta_i)^{-2}\)). We require \(K\) different
distributions so that an arm sampling arm \(i\) very often for some
reason will not have a smaller lower bound by chance.

For each of the \(K\) arms, we introduce an alternative model of the
following kind. Before the start of the algorithm, we introduce a slight
variation in the setup. The idea is similar to how one can switch arms
in the pure exploration setup for multi-armed bandits (compare Audibert
et al., 2010; Garivier \& Kaufmann, 2016) to ``confuse'' any algorithm
without introducing additional complexity to the specified problem.
However, since we do not compare arms directly with each other in the
thresholding bandit problem (switching arms would still keep all means
above the threshold), we instead need to flip an arm \(i\) with respect
to the threshold in order to get the alternative model
\(\mathcal{B}_i\). Since the threshold is \(\tau = 0\), we have for the
flipped arm the new distribution \(\nu_i'\) as defined above. Indeed, we
can define the new product distribution
\(B^i = \nu_1^i \otimes \dots \otimes \nu_K^i\) where for
\(k \leq K, \nu_k' := \nu_i \mathbbm{1}_{k \neq i} + \nu_i' \mathbbm{1}_{k=i}\).
This flipping of the arm corresponds to multiplying all random variables
\(X_{i,t}\) of arm \(i\) by \(-1\).

As in Locatelli et al. (2016), we let for \(i \leq K\),
\(\mathbb{P}_{\mathcal{B}^i}\) define the probability distribution
describing all samples that a bandit strategy can potentially collect up
to horizon \(T\), i.e.~according to the samples
\((X_{k,s})_{k\leq K, s \leq T} \sim (\mathcal{B}^i)^{\otimes T}\).
\((T_k)_{k\leq K}\) denotes the number of samples collected on arm \(k\)
until time \(T\).

\paragraph{Reminder about the goal of the
proof}\label{reminder-about-the-goal-of-the-proof}

At this point, we have defined a finite set of problems of Gaussian
distributions with fixed variance 1 for a given set of set of gaps
between the means of the distributions and the threshold, \((d_k)_k\).
We will now proof a lower bound on the worst case regret of any
algorithm; that is, the maximum error probability that even the best
algorithm makes on this problem.

\paragraph{Statistical Decision}\label{statistical-decision}

Given our formulation of the regret, the expected regret
\(\mathbb{E}[\mathcal{L}(T)]\) corresponds directly to the probability
of making a wrong classification of any of the \(K\) means. In general,
let \(g\) be a binary decision function, and define
\(\Omega = \{g = 0\}\) and \(\Omega^C = \{g = 1\}\). Assume that the
decision \(T\) distinguishes between two states of the world, \(H_0\)
and \(H_1\). Then our overall error probability is given by
\(P_{H_0}(g=1) + P_{H_1}(g = 0) = P_{H_0}(\Omega^C) + P_{H_1}(\Omega)\).

In our case, \(H_0\) may correspond to the state in which we have
\(\mathcal{B}^0\), and \(H_1\) corresponds to the case where any arm has
been flipped, \(\mathcal{B}^i\). Furthermore, we consider the events in
which the algorithm, making the decision, classifies arm \(i\) as being
above the threshold: \(\mathcal{A}_i = \{i \in \hat{S}_\tau\}\). The
event on which all arms are being classified as above the threshold is
consequently given by \(\mathcal{A} = \bigcap \mathcal{A}_i\). However,
Theorem 1 does not bound the overall regret, but the maximum error
probability on the \(K+1\) different models. We can write this now as

\begin{align}
\max_{i \in \{0, \dots, K\}} \mathbb{E}_{\mathcal{B}^i} (\mathcal{L}(T)) & \geq \max \big( \max_{i \in \{1, \dots, K\}} \mathbb{P}_{\mathcal{B}^i}(\mathcal{A}_i), \mathbb{P}_{\mathcal{B}^0}(\mathcal{A}^C) \big) \\
& = \max \big( \max_{i \in \{1, \dots, K\}} \mathbb{P}_{\mathcal{B}^i}(\mathcal{A}_i), 1 - \mathbb{P}_{\mathcal{B}^0}(\mathcal{A}) \big) \label{LocatelliTheorem1ExpRegret}
\end{align}

In order to give a bound on above equation, we start by considering
\(\mathbb{P}_{\mathcal{B}^i}(\mathcal{A}_i)\). This probability can be
written as the expected change in the likelihood of observed data when
moving from distribution \(P_{\mathcal{B}_0}\) to \(P_{\mathcal{B}_i}\),
times the probability under the original state. In other words, the
probability of event \(\mathcal{A}_i\) under state \(\mathcal{B}^i\) is
given as the probability of \(\mathcal{A}_i\) under the (original) state
\(\mathcal{B}^0\) multiplied by a factor. This factor describes whether
the new distribution increases or decreases the probability of the
event, and is given by the change in the data likelihoods due to moving
from \(\mathcal{B}^0\) to \(\mathcal{B}^i\), expressed as the expected
likelihood ratio under \(\mathcal{B}^0\). By definition, the latter is
equal to the empirical Kullback-Leibler divergence of the two
distributions.

We will define the change of distribution in terms of the empirical
log-likelihood for our case. Then, we will show how to replace the
log-likelihood by the empirical Kullback-Leibler divergence. So let us
first introduce the empirical Kullback-Leibler divergence.

Let every \(\nu_i, \nu_i'\) be dominated by the Lebesgue measure
\(\lambda\). Then their density functions are given by
\(f_i = \frac{\der \nu_i}{\der \lambda}\) and
\(f_i' = \frac{\der \nu_i'}{\der \lambda}\) respectively. For
\(T \geq t \geq 0\), define the empirical Kullback-Leibler divergence as

\begin{align*}
\hat{\KL}_{k,t} & = \frac{1}{t} \sum_{s=1}^{t} \log(\frac{\der \nu_k'}{\der \nu_k}(X_{k,s})) \\
& = \frac{1}{t} \sum_{s=1}^{t} \log \big(\frac{f_k'(X_{k,s})}{f_k(X_{k,s})} \big) \\
& = \frac{1}{t} \log \big( \prod_{s=1}^{t} \frac{f_k'(X_{k,s})}{f_k(X_{k,s})} \big) \\
& \stackrel{\text{iid}}{=} \frac{1}{t} \log \big( \frac{f_k'(X_{k,1}, \dots,X_{k,t})}{f_k(X_{k,1}, \dots,X_{k,t})} \big) \\
& = \frac{1}{t} \log \big( \text{LR}((X_{k,1}, \dots,X_{k,t}), \nu_k', \nu_k) \big)
\end{align*}

Since we consider Gaussian distributions with variance 1, we have
density functions
\(f_i(x) \propto \exp \big(-\frac{1}{2} (x-\Delta_i)^2\big)\) and
\(f_i'(x) \propto \exp \big(-\frac{1}{2} (x+\Delta_i)^2\big)\). Plugging
this into the above definition of the empirical Kullback-Leibler
divergence, we easily see

\begin{align*}
\hat{\KL}_{k,t} & = \frac{1}{t} \sum_{s=1}^{t} \log \big(\frac{\exp \big(-\frac{1}{2} (X_{k,s}+\Delta_k)^2\big)}{\exp \big(-\frac{1}{2} (X_{k,s}-\Delta_k)^2\big)} \big) \\
& = \frac{1}{t} \sum_{s=1}^{t} \log \big( \exp(-\frac{1}{2} (X_{k,s}+\Delta_k)^2 + \frac{1}{2} (X_{k,s}-\Delta_k)^2) \big) \\
& = - \frac{1}{t} \sum_{s=1}^{t} 2 X_{k,s} \Delta_k
\end{align*}

\paragraph{Change of Distribution}\label{change-of-distribution}

We are now ready to state the change of distribution. As alluded to
above, we consider how the probability changes for the event
\(\mathcal{A}_i\) when we move from \(\mathcal{B}^0\) to
\(\mathcal{B}^i\). When flipping arm \(i\), we only changed the samples
of this arm, that is, the first \(T_i\) observations we got from arm
\(i\) the algorithm took. The change of distribution is then defined as:

\begin{align*}
\mathbb{P}_{\mathcal{B}^i}(\mathcal{A}_i) & = \mathbb{E}_{\mathcal{B}^0} \big[\mathbbm{1}_{\mathcal{A}_i} \text{LR}((X_{i,1}, \dots,X_{i,T_i}), \nu_i', \nu_i) ] \\
& = \mathbb{E}_{\mathcal{B}^0} \big[\mathbbm{1}_{\mathcal{A}_i} \exp (- T_i \hat{\KL}_{i,T_i}) ]
\end{align*}

Obviously, \(\hat{\KL}_{i,T_i}\) depends on the realized samples
\(X_{i,1}, \dots,X_{i,T_i}\). Consequently, we need to find a bound on
\(\hat{\KL}_{i,T_i}\) itself to incorporate the uncertainty involved in
this metric.

\paragraph{Concentration of the Empirical Kullback-Leibler
Divergence}\label{concentration-of-the-empirical-kullback-leibler-divergence}

For our problem, the Kullback-Leibler divergence of \(\nu_k'\) from
\(\nu_k\) is given by \(\KL_k := KL(\nu_k', \nu_k) = 2\Delta_k^2\).

It is easy to see that

\begin{equation*}
|\hat{\KL}_{k,t} - \KL_{k,t}| = |-\frac{2}{t} \Delta_k \sum_{s=1}^{t}(X_{k,s} - \Delta_k)|
\end{equation*}

itself is a normally distributed random variable with mean zero and a
variance decreasing in \(t\). Thus, one can show as in Locatelli et al.
(2016) that \(\mathbb{P}_{\mathcal{B}^i}(\xi) \geq 3/4\) for the event

\begin{equation}
\xi = \{ \forall k \leq K, \forall t \leq T, |\hat{\KL}_{k,t} - \KL_{k,t}| \leq 4 \Delta_k \sqrt{\frac{\log(4(\log(T)+1)K)}{t}}\}. \label{LocatelliTheorem1EventXi}
\end{equation}

Thus, in order to bound the empirical Kullback-Leibler divergence in our
change of measure, we can consider the intersection of \(\mathcal{A}_i\)
and \(\xi\) as follows and plug in the bound on event \(\xi\):

\begin{align*}
\mathbb{P}_{\mathcal{B}^i}(\mathcal{A}_i) & = \mathbb{E}_{\mathcal{B}^0} \big[\mathbbm{1}_{\mathcal{A}_i} \exp (- T_i \hat{\KL}_{i,T_i}) ] \\
& \geq \mathbb{E}_{\mathcal{B}^0} \big[\mathbbm{1}_{\mathcal{A}_i \cap \xi} \exp (- T_i \hat{\KL}_{i,T_i}) ] \\
& \geq \mathbb{E}_{\mathcal{B}^0} \big[\mathbbm{1}_{\mathcal{A}_i \cap \xi} \exp (- 2 \Delta_i^2 T_i - 4 \Delta_i \sqrt{T_i} \sqrt{\log(4(\log(T)+1)K)}) ]
\end{align*}

Two things are left to do: With \(T_i\), we still have one random
variable left which we cannot leave in our bound. Second, we are
actually not interested in \(\mathbb{P}_{\mathcal{B}^i}(\mathcal{A}_i)\)
but in
\(\max_{i \in \{1,\dots,K\}} \mathbb{P}_{\mathcal{B}^i}(\mathcal{A}_i)\).

\paragraph{\texorpdfstring{Bounding \(T_i\) for
\(\max_{i \in \{1,\dots,K\}}\mathbb{P}_{\mathcal{B}^i}(\mathcal{A}_i)\)}{Bounding T\_i for \textbackslash{}max\_\{i \textbackslash{}in \textbackslash{}\{1,\textbackslash{}dots,K\textbackslash{}\}\}\textbackslash{}mathbb\{P\}\_\{\textbackslash{}mathcal\{B\}\^{}i\}(\textbackslash{}mathcal\{A\}\_i)}}\label{bounding-t_i-for-max_i-in-1dotskmathbbp_mathcalbimathcala_i}

Part of our bound is
\(\max_{i \in \{1,\dots,K\}} \mathbb{P}_{\mathcal{B}^i}(\mathcal{A}_i)\),
so let's continue with that. In general it holds that
\(\max_{1 \leq i \leq K} (X_1, \dots, X_K) \geq \frac{1}{K}\sum_{i=1}^K X_i\).
Thus we can write:

\begin{align*}
\max_{i \in \{1,\dots,K\}} \mathbb{P}_{\mathcal{B}^i}(\mathcal{A}_i) & \geq \frac{1}{K} \sum_{i=1}^{K} \mathbb{P}_{\mathcal{B}^i}(\mathcal{A}_i) \\
& \geq \frac{1}{K} \sum_{i=1}^{K} \mathbb{P}_{\mathcal{B}^i}(\mathcal{A}_i \cap \xi) \\
& \geq \frac{1}{K} \sum_{i=1}^{K} \mathbb{E}_{\mathcal{B}^0} \big[\mathbbm{1}_{\mathcal{A}_i \cap \xi} \exp (- 2 \Delta_i^2 T_i - 4 \Delta_i \sqrt{T_i} \sqrt{\log(4(\log(T)+1)K)}) ]
\end{align*}

where the second line is again the intersection of \(\mathcal{A}_i\) and
\(\xi\) as before in order to apply the change of distribution in the
third line.

What is left to show before we have our final bound is a bound on
\(T_i\). To show that any algorithm has a certain bound on its error, it
only makes sense that every arm has to pulled at least a certain amount
of times. Consequently, no arm should be pulled exclusively.

First, let \(a = \Delta_i \sqrt{T_i}\) and
\(b = 4\sqrt{\log((4\log(T)+1)K)}\). Then use \(ab \leq a^2 + b^2\) to
write \(ab \leq \Delta_i^2 T_i + 16\log((4\log(T)+1)K)\). And so:

\begin{align*}
\max_{i \in \{1,\dots,K\}} \mathbb{P}_{\mathcal{B}^i}(\mathcal{A}_i) & \geq \frac{1}{K} \sum_{i=1}{K} \mathbb{E}_{\mathcal{B}^0} \big[\mathbbm{1}_{\mathcal{A}_i \cap \xi} \exp (- 3 \Delta_i^2 T_i - 16 \log(4(\log(T)+1)K)) ]
\end{align*}

Furthermore, we know that the sum of the individual arm pulls has to be
equal to the overall number of pulls (equal to the budget in the fixed
budget sense), \(\sum_i T_i = T\), and furthermore we know that
\(T_i>0\). The latter holds since any reasonable algorithm has to check
each arm at least once in our setup. This is because we need to check
each arm's performance in comparison to the threshold. Thus it is easy
to show that
\(\exists i: T_i \leq \frac{T}{H \Delta_i^2} = \frac{T}{(\sum_{i=1}^{K} \Delta_i^{-2}) \Delta_i^2}\).
To see this, assume the opposite:
\(\forall i: T_i > T \frac{\sum_{i=1}^K \Delta_i^2}{\Delta_i^2}\). This
implies however
\(\sum_{i=1}^K T_i > \sum_{i=1}^K T \frac{\sum_{i=1}^K \Delta_i^2}{\Delta_i^2} = T \cdot 1 = T\)
which is a contradiction.

So we can write \((\Delta_i \sqrt{T_i})^2 \leq \frac{T}{H}\). We use
\(\mathcal{A} = \cap_{i=1}^K \mathcal{A}_i\) to write:

\begin{align*}
\max_{i \in \{1,\dots,K\}} \mathbb{P}_{\mathcal{B}^i}(\mathcal{A}_i) & \geq \frac{1}{K} \sum_{i=1}^{K} \mathbb{E}_{\mathcal{B}^0} \big[\mathbbm{1}_{\mathcal{A}_i \cap \xi} \exp (- 3 \Delta_i^2 T_i - 16 \log(4(\log(T)+1)K)) \big] \\
& = \mathbb{E}_{\mathcal{B}^0} \big[\mathbbm{1}_{\mathcal{A} \cap \xi} \frac{1}{K} \sum_{i=1}{K} \exp (- 3 \Delta_i^2 T_i - 16 \log(4(\log(T)+1)K)) \big]
\end{align*}

We write
\(\frac{1}{K} \sum_{i=1}^K \exp(-3T_i \Delta_i^2) \geq \frac{1}{K} \sum_{i=1}^K \exp(-3 T/H) = \exp(-3 T/H)\).
Thus:

\begin{align*}
\max_{i \in \{1,\dots,K\}} \mathbb{P}_{\mathcal{B}^i}(\mathcal{A}_i) & = \mathbb{E}_{\mathcal{B}^0} \big[\mathbbm{1}_{\mathcal{A} \cap \xi} \frac{1}{K} \sum_{i=1}^{K} \exp (- 3 \Delta_i^2 T_i) \exp(-16 \log(4(\log(T)+1)K)) ] \\
& \geq \mathbb{E}_{\mathcal{B}^0} \big[\mathbbm{1}_{\mathcal{A} \cap \xi} \frac{1}{K} \sum_{i=1}^{K} \exp (- 3 \frac{T}{H}) \exp(-16 \log(4(\log(T)+1)K)) ] \\
& = \mathbb{E}_{\mathcal{B}^0} \big[\mathbbm{1}_{\mathcal{A} \cap \xi} \exp (- 3 \frac{T}{H}) \exp(-16 \log(4(\log(T)+1)K)) ] \\
& = \mathbb{P}_{\mathcal{B}^0} (\mathcal{A} \cap \xi) \exp (- 3 \frac{T}{H} -16 \log(4(\log(T)+1)K))
\end{align*}

\paragraph{Bringing everything
together}\label{bringing-everything-together}

Consider again the risk which we need to bound from equation
\eqref{LocatelliTheorem1ExpRegret}. We bound the maximum expected loss
over all \(\mathcal{B}_i\). By again using the fact that in general
\(\max_{1 \leq i \leq K} (X_1, \dots, X_K) \geq \frac{1}{K} \sum_{i = 1}^K X_i\),
we plug in our previous calculations:

\begin{align}
\max_{i \in \{0, \dots, K\}} \mathbb{E}_{\mathcal{B}^i} (\mathcal{L}(T)) & \geq \max \big( \max_{i \in \{1, \dots, K\}} \mathbb{P}_{\mathcal{B}^i}(\mathcal{A}_i), 1 - \mathbb{P}_{\mathcal{B}^0}(\mathcal{A}) \big) \\
& \geq \frac{1}{2}\mathbb{P}_{\mathcal{B}^0} (\mathcal{A} \cap \xi) \exp (- 3 \frac{T}{H} -16 \log(4(\log(T)+1)K)) + \frac{1}{2}(1 - \mathbb{P}_{\mathcal{B}^0}(\mathcal{A})) \label{LocatelliTheorem1DefinitionOfRisk}
\end{align}

From \eqref{LocatelliTheorem1EventXi} we know that for any
\(i \in \{0, \dots, K\}\), \(\mathbb{P}_{\mathcal{B}^i}(\xi) \geq 3/4\).
Following the argument in Locatelli et al. (2016), we now consider two
cases: \(\mathbb{P}_{\mathcal{B}^0}(\mathcal{A}) \geq 1/2\) and
\(\mathbb{P}_{\mathcal{B}^0}(\mathcal{A}) \leq 1/2\). In the former
case, we can plug in the two probabilities into
\eqref{LocatelliTheorem1DefinitionOfRisk}. If
\(\mathbb{P}_{\mathcal{B}^0}(\mathcal{A}) \geq 1/2\) and
\(\mathbb{P}_{\mathcal{B}^0}(\xi) \geq 3/4\), then
\(\mathbb{P}_{\mathcal{B}^0}(\mathcal{A} \cap \xi) \geq 1/4\). Thus:

\begin{align*}
\max_{i \in \{0, \dots, K\}} \mathbb{E}_{\mathcal{B}^i} (\mathcal{L}(T)) & \geq \max \big( \max_{i \in \{1, \dots, K\}} \mathbb{P}_{\mathcal{B}^i}(\mathcal{A}_i), 1 - \mathbb{P}_{\mathcal{B}^0}(\mathcal{A}) \big) \\
& \geq \frac{1}{2}\mathbb{P}_{\mathcal{B}^0} (\mathcal{A} \cap \xi) \exp (- 3 \frac{T}{H} -16 \log(4(\log(T)+1)K)) + \frac{1}{2}(1 - \mathbb{P}_{\mathcal{B}^0}(\mathcal{A})) \\
& \geq \frac{1}{8} \exp (- 3 \frac{T}{H} -16 \log(4(\log(T)+1)K))
\end{align*}

where we drop the second summand of the second line so that the
inequality holds.

On the other hand, when
\(\mathbb{P}_{\mathcal{B}^0}(\mathcal{A}) \leq 1/2\), then we know that
\(\mathbb{P}_{\mathcal{B}^0}(\mathcal{A} \cap \xi) \leq 1/2\) for any
\(\mathbb{P}_{\mathcal{B}^0}(\xi)\). At the same time
\(1-\mathbb{P}_{\mathcal{B}^0}(\mathcal{A}) = \mathbb{P}_{\mathcal{B}^0}(\mathcal{A^C}) \geq 1/2\).
Consequently,
\(1/4 \leq \frac{1}{2}(1 - \mathbb{P}_{\mathcal{B}^0}(\mathcal{A})) \leq 1/2\),
while
\(\frac{1}{2}\mathbb{P}_{\mathcal{B}^0} (\mathcal{A} \cap \xi) \exp (- 3 \frac{T}{H} -16 \log(4(\log(T)+1)K)) \geq 0\).
And so the stronger lower bound in that case is, as noticed by Locatelli
et al. (2016), given simply by

\begin{align*}
\max_{i \in \{0, \dots, K\}} \mathbb{E}_{\mathcal{B}^i} (\mathcal{L}(T)) & \geq \max \big( \max_{i \in \{1, \dots, K\}} \mathbb{P}_{\mathcal{B}^i}(\mathcal{A}_i), 1 - \mathbb{P}_{\mathcal{B}^0}(\mathcal{A}) \big) \\
& \geq 1 - \mathbb{P}_{\mathcal{B}^0}(\mathcal{A}) = \mathbb{P}_{\mathcal{B}^0}(\mathcal{A^C}) \\
& \geq \frac{1}{2}
\end{align*}


\end{document}
