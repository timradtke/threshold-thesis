\documentclass[12pt,]{article}
\usepackage{lmodern}
\usepackage{amssymb,amsmath}
\usepackage{ifxetex,ifluatex}
\usepackage{fixltx2e} % provides \textsubscript
\ifnum 0\ifxetex 1\fi\ifluatex 1\fi=0 % if pdftex
  \usepackage[T1]{fontenc}
  \usepackage[utf8]{inputenc}
\else % if luatex or xelatex
  \ifxetex
    \usepackage{mathspec}
  \else
    \usepackage{fontspec}
  \fi
  \defaultfontfeatures{Ligatures=TeX,Scale=MatchLowercase}
\fi
% use upquote if available, for straight quotes in verbatim environments
\IfFileExists{upquote.sty}{\usepackage{upquote}}{}
% use microtype if available
\IfFileExists{microtype.sty}{%
\usepackage{microtype}
\UseMicrotypeSet[protrusion]{basicmath} % disable protrusion for tt fonts
}{}
\usepackage[margin=1in]{geometry}
\usepackage{hyperref}
\PassOptionsToPackage{usenames,dvipsnames}{color} % color is loaded by hyperref
\hypersetup{unicode=true,
            pdftitle={Theorems and Proofs},
            pdfauthor={Tim Radtke},
            colorlinks=true,
            linkcolor=Maroon,
            citecolor=Blue,
            urlcolor=blue,
            breaklinks=true}
\urlstyle{same}  % don't use monospace font for urls
\usepackage{graphicx,grffile}
\makeatletter
\def\maxwidth{\ifdim\Gin@nat@width>\linewidth\linewidth\else\Gin@nat@width\fi}
\def\maxheight{\ifdim\Gin@nat@height>\textheight\textheight\else\Gin@nat@height\fi}
\makeatother
% Scale images if necessary, so that they will not overflow the page
% margins by default, and it is still possible to overwrite the defaults
% using explicit options in \includegraphics[width, height, ...]{}
\setkeys{Gin}{width=\maxwidth,height=\maxheight,keepaspectratio}
\IfFileExists{parskip.sty}{%
\usepackage{parskip}
}{% else
\setlength{\parindent}{0pt}
\setlength{\parskip}{6pt plus 2pt minus 1pt}
}
\setlength{\emergencystretch}{3em}  % prevent overfull lines
\providecommand{\tightlist}{%
  \setlength{\itemsep}{0pt}\setlength{\parskip}{0pt}}
\setcounter{secnumdepth}{5}
% Redefines (sub)paragraphs to behave more like sections
\ifx\paragraph\undefined\else
\let\oldparagraph\paragraph
\renewcommand{\paragraph}[1]{\oldparagraph{#1}\mbox{}}
\fi
\ifx\subparagraph\undefined\else
\let\oldsubparagraph\subparagraph
\renewcommand{\subparagraph}[1]{\oldsubparagraph{#1}\mbox{}}
\fi

%%% Use protect on footnotes to avoid problems with footnotes in titles
\let\rmarkdownfootnote\footnote%
\def\footnote{\protect\rmarkdownfootnote}

%%% Change title format to be more compact
\usepackage{titling}

% Create subtitle command for use in maketitle
\newcommand{\subtitle}[1]{
  \posttitle{
    \begin{center}\large#1\end{center}
    }
}

\setlength{\droptitle}{-2em}
  \title{Theorems and Proofs}
  \pretitle{\vspace{\droptitle}\centering\huge}
  \posttitle{\par}
  \author{Tim Radtke}
  \preauthor{\centering\large\emph}
  \postauthor{\par}
  \predate{\centering\large\emph}
  \postdate{\par}
  \date{4/23/2017}

\usepackage[retainorgcmds]{IEEEtrantools}
\usepackage{bm}
\usepackage{amsmath}
\usepackage{bbm}
\newtheorem{theorem}{Theorem}
\newtheorem{lemma}{Lemma}
\newcommand{\KL}{\,\text{KL}}
\newcommand{\der}{\,\text{d}}

\begin{document}
\maketitle

\subsection{Theorem 2}\label{theorem-2}

\begin{theorem}[Locatelli et al., 2016] \label{theorem:LocatelliTheorem4}
Let $K \geq 0$, $T \geq 2K$, and consider a problem $\mathbb{B}$. Assume that
all arms $\nu_k$ of the problem are R-sub-Gaussian with means $\mu_k$. Let $\tau
\in \mathbb{R}, \epsilon \geq 0$.

Algorithm APT's expected loss is upper bounded on this problem as 
\begin{equation*} \mathbb{E}(\mathcal{L}(T)) \leq \exp
(-\frac{1}{64R^2}\frac{T}{H} + 2 \log (T) + 1)K)) \end{equation*}

where we remind that $H = \sum_i (|\mu_i - \tau | + \epsilon)^{-2}$ and where
$\mathbb{E}$ is the expectation according to the samples of the problem.
\end{theorem}

\emph{Proof of Theorem \ref{theorem:LocatelliTheorem4}:} To prove above
theorem, we first define a favorable event \(\xi\) on which we hope to
make a loss with only a small probability. Indeed, it is the event for
which we want to upper bound the expected loss. As we want to find an
upper bound for the expected loss of the APT algorithm, the event
corresponds to the decision made by it. Given that the APT algorithm's
sampling rule and final decision is based on the empirical mean of each
arm, it will have low expected regret, if it finds empirical mean
estimates \(\hat{\mu}_i(t) = \frac{1}{s} \sum_{t=1}^s X_{i,t}\) that are
close to the true means \(\mu_i\) of the arms. Then, we would expect
arms with \(\hat{\mu}_i(T) > \tau + \epsilon\) to be classified
correctly, and arms with \(\hat{\mu}_i(T) < \tau - \epsilon\) to be
rejected.

Indeed, we have
\(\mathbb{E}(\mathcal{L}(T)) = P(S_{\tau + \epsilon} \cap \hat{S}_\tau^C \neq \emptyset \lor S_{\tau-\epsilon}^C \cap \hat{S}_{\tau} \neq \emptyset)\).
But this just corresponds to
\(P(\{\exists i: ((\mu_i < \tau-\epsilon) \land(\hat{\mu}_i \geq \tau)) \lor ((\mu_i > \tau + \epsilon) \land (\hat{\mu}_i < \tau)))\).
Each of the two \((...)\lor(...)\) conditions corresponds to
\(| \hat{\mu}_i - \mu_i | \geq \epsilon\). It follows that the expected
loss can be rewritten as:
\(\mathbb{E}(\mathcal{L}(T)) = P(\exists i, \exists s: |\hat{\mu}_{i,s} - \mu_i | \geq \epsilon) = P(\xi^C)\).
This is exactly the probability of classifying at least one arm
incorrectly which results in a loss of size 1.

Thus we define as the favorable event as:

\begin{equation} 
\xi = \{\forall i, \forall s: | \hat{\mu}_{i,s} - \mu_i | \leq
\epsilon\} 
\end{equation}

Furthermore, if we write
\(\xi_{i,s} = \{ |\hat{\mu}_{i,s}-\mu_i | \leq \epsilon \}\), then the
following holds by a union bound.

\begin{equation} 
P(\xi) = P(\cap_i \cap_s \xi_{i,s}) = 1 - P(\cup_i \cup_s
\xi_{i,s}^C) \geq 1 - \sum_i \sum_s P(| \hat{\mu}_{i,s} - \mu_i| \geq \epsilon)
\label{theorem2_prob_fav_event} 
\end{equation}

Furthermore, for any \(i \in \{1, \dots, K\}\), for any
\(s \in \{1, \dots, T(i)\}\), with \(X_{it}\) being the random sample
from the R-sub-Gaussian distribution \(v_i\) at time \(t\):

\begin{equation} 
P(| \hat{\mu}_{i,s} - \mu_i| \geq \epsilon) = P(| (\frac{1}{s}
\sum_{t=1}^{s} X_{it} - \mu_i) | \geq \epsilon) = P(| \sum_{t=1}^{s} (X_{it} -
\mu_i) | \geq s \epsilon) 
\end{equation}

where the latter can be bounded by a Hoeffding inequality:

\begin{equation} 
P(| \sum_{t=1}^{s} (X_{it} - \mu_i) | \geq s \epsilon) \leq
2\exp (-\frac{\epsilon^2 s}{2 R^2}) \label{theorem2_hoeffding} 
\end{equation}

The Hoeffding inequality as used here is as follows:

\begin{lemma}[Hoeffding's Inequality] \label{lemma:HoeffdingInequality}
Consider $X_1, \dots, X_T$ independent random variables from R-sub-Gaussian distributions $\nu_i$ with mean $\mu_i$. Then, for any $\epsilon > 0$,

\begin{equation*}
\mathbb{P} \Big( | \sum_{i = 1}^T (X_i - \mu_i) | \geq \epsilon \Big) \leq 2\exp \Big(\frac{-\epsilon^2}{2T R^2}\Big).
\end{equation*}
\end{lemma}

Plugging equation \eqref{theorem2_hoeffding} back into equation
\eqref{theorem2_prob_fav_event}, we get:

\begin{equation}
P(\xi) \geq 1 - \sum_i \sum_s P(| \hat{\mu}_{i,s} - \mu_i| \geq \epsilon) \geq 1 - 2 \sum_i \sum_s \exp (-\frac{\epsilon^2 s}{2 R^2})
\end{equation}

Now, in order to match the performance of the lower bound on the
expected loss derived in Theorem \ref{theorem:Locatelli2016Theorem1}, we
will choose \(\epsilon = \sqrt{\frac{T \delta^2}{Hs}}\). Thus:

\begin{equation}
P(\xi) \geq 1 - 2 \sum_i \sum_s \exp (-\frac{\epsilon^2 s}{2 R^2}) = 1 - 2 \sum_i \sum_s \exp (-\frac{T \delta^2}{2 R^2 H}) \geq 1 - TK \exp(-\frac{T\delta^2}{2R^2H})
\end{equation}

Thus, we do not exactly get the bound from Locatelli et al. (2016)
because we used the Hoeffding inequality instead of the martingale
bound; but the the method should be clear.

What is left now is to show that the favorable event still holds under
the proposed \(\epsilon\), that is, we need to show that under
\(P(\xi) \geq 1 - TK \exp(-\frac{T\delta^2}{2R^2H})\) we still reject
arms that are under \(\tau - \epsilon\) and accept those above
\(\tau + \epsilon\).

\paragraph{Step 2: Lower Bound on Number of Pulls of Difficult
Arm}\label{step-2-lower-bound-on-number-of-pulls-of-difficult-arm}

In order to see that arms are correctly classified with large
probability, we need to show that they are are pulled sufficiently
often, that is, we have a degree of exploration for each arm \(i\) that
allows us to identify its characteristics. To see this, rewrite the
favorable event of a single arm as follows with \(s = T(i)\):

\begin{equation}
\xi_{i,s} = \{ |\hat{\mu}_{i,s}-\mu_i | \leq \epsilon \} = \{ |\hat{\mu}_{i,s} - \mu_i | \leq \sqrt{\frac{T \delta^2}{H T(i)}}\}
\end{equation}

To make sure that this deviation is small enough, we need to show that
at time \(T\), all \(T(i)\) have a sufficient lower bound and thus all
arms \(i\) have been explored sufficiently to classify correctly with
probability larger \(TK \exp(-\frac{T\delta^2}{2R^2H})\).

To start off, notice that due to the initialization of the APT
algorithm, every \(T_i(T) \geq 1\). Furthermore, we have assumed that
\(T>2K\). Now, at time \(T\), consider an arm \(k\) that has been pulled
\(T_k(T) -1 \geq \frac{(T-K)}{H \Delta^2_k}\) times. That is, it has
been pulled at least proportionally to it's contribution to the overall
hardness of the problem \(H\). Recall
\(H = \sum_{i=1}^K(|\mu_i - \tau|+\epsilon)^-2\) and
\(\delta^2_i = (|\mu_i - \tau | + \epsilon)^2\). To see that such arm
\(k\) exists, assume it does not. Then

\begin{equation*}
T_i(T) - 1 < \frac{T-K}{H \Delta_i^2} \forall i
\end{equation*}\begin{equation*}
T- K = \sum_{i=1}^{K} T_i(T) - 1 < \sum_{i=1}^{K} \frac{T-K}{H \Delta_i^2} = \frac{T-K}{H \Delta_i^2} \sum_{i=1}^{K} \frac{1}{\Delta_i^2} = T-K
\end{equation*}

which is a contradiction.

Consequently, using the assumption \(T>2K\) from above, we have

\begin{equation*}
T_k(T) \geq T_k(T) - 1 \geq \frac{T-K}{H \Delta^2_k} = \frac{T}{2H\Delta_k^2}
\end{equation*}

This states a lower bound on the amount of pulls \(T_k(t)\) for an arm
\(k\) that contributes more than average to the overall hardness of the
problem.

\paragraph{Step 3: Lower Bound on Number of Pulls of Simple
Arm}\label{step-3-lower-bound-on-number-of-pulls-of-simple-arm}

Next, we would like to derive bounds on the number of pulls of the
remaining arms \(i\) at the time \(t\), the last pull of arm \(k\). To
do so, consider that we are on the favorable event \(\xi\). For every
arm \(i\), it holds:

\begin{equation}
|\hat{\mu}_i(t) - \mu_i | \leq \sqrt{\frac{T \delta^2}{HT_i(t)}} \label{step3_favorable_event} 
\end{equation}

As described in Locatelli et al., using the reverse triangle inequality,
we can write

\begin{IEEEeqnarray*}{rCl}
|\hat{\mu}_i(t) - \mu_i | & = & | (\hat{\mu}_i(t) - \tau) - (\mu_i-\tau) |
\\
& \geq & | |\hat{\mu}_i(t) - \tau| - |\mu_i-\tau| |
\\
& = & | (|\hat{\mu}_i(t) - \tau| + \epsilon) - (|\mu_i-\tau| - \epsilon) |
\\
& = & | \hat{\Delta}_i(t) - \Delta_i |
\end{IEEEeqnarray*}

Consequently, we can again employ the hardness of the task to start
deriving bounds on the number of pulls of the arms:

\begin{equation*}
| \hat{\Delta}_i(t) - \Delta_i | \leq |\hat{\mu}_i(t) - \mu_i | \sqrt{\frac{T \delta^2}{HT_i(t)}}
\end{equation*}

From this, we get

\begin{IEEEeqnarray*}{rlCrl}
\Delta_i - \sqrt{\frac{T \delta^2}{HT_i(t)}} & \leq & \hat{\Delta}_i(t) & \leq & \Delta_i + \sqrt{\frac{T \delta^2}{HT_i(t)}}
\end{IEEEeqnarray*}

and in particular for the arm \(k\) discussed above

\begin{IEEEeqnarray}{rlCrl}
\Delta_k - \sqrt{\frac{T \delta^2}{HT_k(t)}} & \leq & \hat{\Delta}_k(t) & \leq & \Delta_k + \sqrt{\frac{T \delta^2}{HT_k(t)}} \label{delta_hat_k_bounds}
\end{IEEEeqnarray}

To bound \(T_i(t)\), it is helpful to compare \(\Delta_i\) and
\(\Delta_k\) at time \(t\). From the definition of the APT algorithm, we
know that since arm \(k\) has been pulled at time \(t\) (``Pull arm
\(I_{t+1} = \arg \min_i \mathcal{B}_i (t+1)\)''):

\begin{IEEEeqnarray*}{rCl}
\mathcal{B}_k(t) & \leq & \mathcal{B}_i(t)
\\
\sqrt{T_k(t)} \hat{\Delta}_k(t) & \leq & \sqrt{T_i(t)} \hat{\Delta}_i(t)
\end{IEEEeqnarray*}

Thus, every arm \(i\) has a lower bound on it's number of pulls given by
the lower bound for arm \(k\) proportional with it's inverse relative
estimated hardness \(\hat{\Delta}_k(t) / \hat{\Delta}_i(t)\). We can
easily derive a lower bound for the left hand side by plugging in
\eqref{delta_hat_k_bounds}:

\begin{IEEEeqnarray}{rCl}
\sqrt{T_k(t)} (\Delta_k - \sqrt{\frac{T \delta^2}{HT_k(t)}}) & \leq & \mathcal{B}_k(t) 
\\
\sqrt{\frac{T}{2H\Delta_k^2}} (\Delta_k - \sqrt{2}\Delta_k \delta) & = & \sqrt{\frac{T}{H}} (\frac{1}{\sqrt{2}} - \delta)
\\
& \leq & \mathcal{B}_k(t)
\end{IEEEeqnarray}

since \(T_k(t) \geq T/(2H\Delta^2_k)\) and
\(\sqrt{T \delta^2 / (HT_k(t))} \geq \sqrt{T \delta^2 / (H\frac{T}{2H \Delta^2_k})} = \sqrt{2}\Delta_k \delta\).

Next, upper bound \(\mathcal{B}_i(t)\). In contrast to \(T_k(t)\), there
is no lower bound for \(T_i(t)\) available yet that we could plug in.
Thus we derive:

\begin{IEEEeqnarray*}{rCl}
\mathcal{B}_i(t) & = & \sqrt{T_i(t)} \hat{\Delta}_i(t) 
\\
& \leq & \sqrt{T_i(t)} (\Delta_i + \sqrt{\frac{T \delta^2}{H T_i(t)}})
\\
& = & \sqrt{T_i(t)} \Delta_i + \delta \sqrt{\frac{T}{H}}
\end{IEEEeqnarray*}

Combining the two bounds, we get a lower bound on the pulls for every
other arm \(i\):

\begin{IEEEeqnarray*}{rCl}
\Delta_i \sqrt{T_i(t)} + \delta \sqrt{\frac{T}{H}} & \geq & (\frac{1}{\sqrt{2}} - \delta) \sqrt{\frac{T}{H}}
\\
\sqrt{T_i(t)} & \geq & (\frac{1}{\sqrt{2}} - 2\delta) \sqrt{\frac{T}{H}} \frac{1}{\Delta_i}
\\ 
& & \text{(choose $2\delta > 1 / \sqrt{2}$ such that RHS is greater 0)}
% \\
% T_i(t) & \geq & (\frac{1}{2} - 2\sqrt{2} \delta + 4\delta^2) \underbrace{\frac{T}{H} \frac{1}{\Delta_i^2}}_\text{$i$'s share of the budget}
% \\
% T_i(t) & \geq & (1 - 4 \sqrt{2} \delta + 8\delta^2) \frac{T}{2H} \frac{1}{\Delta_i^2}
\end{IEEEeqnarray*}

\paragraph{Conclusion}\label{conclusion}

As stated before, we require the lower bound on \(T_i(t)\) in order to
show that
\(|\hat{\mu}_i(t) - \mu_i | \leq \sqrt{frac{T \delta^2}{H T_i(t)}}\)
holds for all \(i\). So if we now simply plug in the derived lower
bound, we get:

\begin{equation*}
\mu_i - \Delta_i (\frac{\sqrt{2}\delta}{1-2\sqrt{2}\delta}) \leq \hat{\mu}_i(t) \leq \mu_i + \Delta_i (\frac{\sqrt{2}\delta}{1-2\sqrt{2}\delta})
\end{equation*}

and with \(\lambda := (\frac{\sqrt{2}\delta}{1-2\sqrt{2}\delta}) > 0\):

\begin{equation*}
\mu_i - \Delta_i \lambda \leq \hat{\mu}_i(t) \leq \mu_i + \Delta_i \lambda
\end{equation*}

Under this formulation on \(\xi\), do we accept arms with
\(\mu_i \geq \tau + \epsilon\)? For these arms holds that
\(\Delta_i = \mu_i - \tau + \epsilon\). Consequently:

\begin{IEEEeqnarray*}{rCl}
\mu_i - (\mu_i - \tau + \epsilon) \lambda & \leq & \hat{\mu}_i(t) 
\\ 
\hat{\mu}_i(t) & \geq & \tau \lambda + \mu (1-\lambda) - \epsilon \lambda
\end{IEEEeqnarray*}

The arms are accepted if

\begin{IEEEeqnarray*}{rCl}
\hat{\mu}_i(t) - \epsilon & \geq & 0
\end{IEEEeqnarray*}

Or equivalently as long as

\begin{IEEEeqnarray*}{rCl}
\tau \lambda + \mu(1-\lambda) - \epsilon \lambda - \tau & \geq & 0
\\
\lambda & \leq & \frac{1}{2}
\end{IEEEeqnarray*}


\end{document}
