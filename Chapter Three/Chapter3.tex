\documentclass[12pt,]{article}
\usepackage{lmodern}
\usepackage{amssymb,amsmath}
\usepackage{ifxetex,ifluatex}
\usepackage{fixltx2e} % provides \textsubscript
\ifnum 0\ifxetex 1\fi\ifluatex 1\fi=0 % if pdftex
  \usepackage[T1]{fontenc}
  \usepackage[utf8]{inputenc}
\else % if luatex or xelatex
  \ifxetex
    \usepackage{mathspec}
  \else
    \usepackage{fontspec}
  \fi
  \defaultfontfeatures{Ligatures=TeX,Scale=MatchLowercase}
\fi
% use upquote if available, for straight quotes in verbatim environments
\IfFileExists{upquote.sty}{\usepackage{upquote}}{}
% use microtype if available
\IfFileExists{microtype.sty}{%
\usepackage{microtype}
\UseMicrotypeSet[protrusion]{basicmath} % disable protrusion for tt fonts
}{}
\usepackage[margin=1in]{geometry}
\usepackage{hyperref}
\PassOptionsToPackage{usenames,dvipsnames}{color} % color is loaded by hyperref
\hypersetup{unicode=true,
            pdftitle={Chapter Three},
            pdfauthor={Tim Radtke},
            colorlinks=true,
            linkcolor=Maroon,
            citecolor=Blue,
            urlcolor=blue,
            breaklinks=true}
\urlstyle{same}  % don't use monospace font for urls
\usepackage{graphicx,grffile}
\makeatletter
\def\maxwidth{\ifdim\Gin@nat@width>\linewidth\linewidth\else\Gin@nat@width\fi}
\def\maxheight{\ifdim\Gin@nat@height>\textheight\textheight\else\Gin@nat@height\fi}
\makeatother
% Scale images if necessary, so that they will not overflow the page
% margins by default, and it is still possible to overwrite the defaults
% using explicit options in \includegraphics[width, height, ...]{}
\setkeys{Gin}{width=\maxwidth,height=\maxheight,keepaspectratio}
\IfFileExists{parskip.sty}{%
\usepackage{parskip}
}{% else
\setlength{\parindent}{0pt}
\setlength{\parskip}{6pt plus 2pt minus 1pt}
}
\setlength{\emergencystretch}{3em}  % prevent overfull lines
\providecommand{\tightlist}{%
  \setlength{\itemsep}{0pt}\setlength{\parskip}{0pt}}
\setcounter{secnumdepth}{5}
% Redefines (sub)paragraphs to behave more like sections
\ifx\paragraph\undefined\else
\let\oldparagraph\paragraph
\renewcommand{\paragraph}[1]{\oldparagraph{#1}\mbox{}}
\fi
\ifx\subparagraph\undefined\else
\let\oldsubparagraph\subparagraph
\renewcommand{\subparagraph}[1]{\oldsubparagraph{#1}\mbox{}}
\fi

%%% Use protect on footnotes to avoid problems with footnotes in titles
\let\rmarkdownfootnote\footnote%
\def\footnote{\protect\rmarkdownfootnote}

%%% Change title format to be more compact
\usepackage{titling}

% Create subtitle command for use in maketitle
\newcommand{\subtitle}[1]{
  \posttitle{
    \begin{center}\large#1\end{center}
    }
}

\setlength{\droptitle}{-2em}
  \title{Chapter Three}
  \pretitle{\vspace{\droptitle}\centering\huge}
  \posttitle{\par}
  \author{Tim Radtke}
  \preauthor{\centering\large\emph}
  \postauthor{\par}
  \predate{\centering\large\emph}
  \postdate{\par}
  \date{5/31/2017}

\usepackage[retainorgcmds]{IEEEtrantools}
\usepackage{bm}
\usepackage{amsmath}
\usepackage{bbm}
\newtheorem{theorem}{Theorem}
\newtheorem{lemma}{Lemma}
\newcommand{\KL}{\,\text{KL}}
\newcommand{\der}{\,\text{d}}

\begin{document}
\maketitle

{
\hypersetup{linkcolor=black}
\setcounter{tocdepth}{3}
\tableofcontents
}
\section{KL and Bayesian Thresholding
Bandits}\label{kl-and-bayesian-thresholding-bandits}

\subsection{Introduction}\label{introduction}

We have introduced the thresholding bandit problem in chapter 1. The APT
algorith described in Locatelli et al. (2016) is able to match a lower
bound (be more specific!) given in the same paper, and has several
favorable characteristics such as its overall simplicity and it being an
anytime algorithm. However, the bounds on the expected loss assume the
arms' distributions to be \(R\)-sub-Gaussian. This is reflected in the
algorithm's complexity measure
\(H = \sum_{i=1}^K (\Delta_i^{\tau, \epsilon})^{-2}\), where
\(\Delta_i^{\tau,\epsilon} = |\mu_i - \tau| + \epsilon\), which -- given
the known threshold \(\tau \pm \epsilon\) -- only depends on the
distribution's mean. This is appropriate for Gaussian arms with equal
variances, but potentially inefficient for other kinds of distributions.

In the case of Gaussian distributions with varying variances, for
example, it makes sense to take the variance into account; after all,
given two Gaussians with equal mean, intuitively we would prefer to get
a sample from the distribution of larger variance.

In the case of Gaussians, the complexity \(H\) is nice also because it
reflects the Kullback-Leibler divergence between two Gaussians. And
since Bernoulli distributions are bounded on the interval \([0,1]\),
they are consequently sub-Gaussian, meaning that their Kullback-Leibler
divergence can be approximated by the Gaussian one. However, while
Bernoullis are sub-Gaussian, the approximation of their divergence by
the Gaussian divergence becomes less and less optimal as the parameter
(and thus mean) goes to \(0\) or \(1\), as shown in chapter 1. And while
online advertisements might not be an obvious use case for thresholding
bandits, they are just one real life scenario dominated by Bernoulli
distributions with very small parameters. Thus it is worthwile to adapt
current solutions of the thresholding bandit problem to cases in which
the Gaussian Kullback-Leibler divergence is no longer an appropriate
approximation.

Large motiviation for the idea of adjusting the complexity measure (and
thus the algorithms) to distributions more generally based on the
Kullback-Leibler divergence stems from the work discussed in chapter 2,
complexity measures were defined for the pure exploration bandits in
terms of the Kullback-Leibler divergence. Based on this, the authors
were for example able to point our differences between Gaussian and
Bernoulli bandits. We expect that results in a similar vain could be
derived for thresholding bandits.

To this end, in the following, we first present an extension of mean
based algorithms to an algorithm that incorporates empirical variances
as described in Mukherjee et al. (2017). Afterwards, we propose and
discuss potential further improvements based on Kullback-Leibler
divergence and Bayesian posterior distributions. Finally, we compare the
performance of different algorithms based on simulations and real world
data sets.

\subsection{Variance Based Algorithms for Thresholding
Bandits}\label{variance-based-algorithms-for-thresholding-bandits}

Mukherjee et al. (2017) propose an algorithm based on variance estimates
called Augmented-UCB (AugUCB). The setting in which the algorithm is
proposed is very similar to the setting in Locatelli et al. (2016). The
algorithm itself, however, differs heavily from the APT algorithm. So is
AugUCB for example not an anytime algorithm. Furthermore, arms can be
deleted in every round if there are ``far enough'' away from the mean.
Both the decision of which arm is played in a given round, and whether
an arm is removed from the active set of arm, is based not only on the
empirical mean estimate \(\mu_i(t)\), but also on a term \(s_i\) that is
increasing linearly in the estimated standard deviation. While the APT
algorithm plays the arm minimizing the quantity
\(B_i(t+1) = \sqrt{T_i(t)} \hat{\Delta}_i(t) = \sqrt{T_i(t)} (|\hat{\mu}_i(t) - \tau| + \epsilon)\),
AugUCB plays the arm minimizing the quantity
\((\hat{\Delta}_i(t) - 2s_i)\), where \(s_i\) is based on the variance
estimate \(\hat{v}_i\), and an exploration \(\psi_m\) term linear in
\(T\), and an elimination parameter \(\rho\) (compare Algorithm 1 in
Mukherjee et al., 2017):

\begin{equation*}
s_i = \sqrt{\frac{\rho \psi_m (\hat{v}_i + 1) \log(T \epsilon_m)}{4n_i}}
\end{equation*}

Consequently, the authors also introduce complexity measures which take
the variance into account. From Gabillon et al. (2011) they use
\(H_{\sigma, 1}= \sum_{i=1}^K \frac{\sigma_i + \sqrt{\sigma_i^2 + (16/3)\Delta_i}}{\Delta_i^2}\).
Furthermore, with
\(\stackrel{\sim}{\Delta}_i^2 = \frac{\Delta_i^2}{\sigma_i + \sqrt{\sigma_i^2 + (16/3)\Delta_i}}\),
and with \(\stackrel{\sim}{\Delta}_{(i)}\) being an increasing ordering
of \(\stackrel{\sim}{\Delta}_i\), the authors define a complexity
corresponding to \(H_{CSAR,2}\) in Chen et al. (2014):

\begin{equation*}
H_{\sigma,2} = \max_{i \ in \mathcal{A}} \frac{i}{\stackrel{\sim}{\Delta}_{(i)}^2}.
\end{equation*}

While the latter complexity appears in the provided upper bound on the
AugUCB algorithm's expected loss, it could be replaced by
\(H_{\sigma, 1}\) since \(H_{\sigma, 2} \leq H_{\sigma,1}\).
\(H_{\sigma,1}\) is of course more directly comparable to the complexity
\(H\) that we are used to.

While in the case of the APT algorithm the number of times a specific
arm has been drawn \(T_i\) measures how concentrated the estimated mean
is around the true mean (and thus for large \(T_i\) an arm is no longer
drawn even if \(\hat{\mu}\) is close to \(\tau\)); in the case of the
AugUCB algorithm this is regulated by the combination of the variance
estimate \(\hat{v}_i\) and the exploration factor \(\psi_m\), where they
ensure that certain arms are no longer pulled when the algorithm is
sufficiently confident in its knowledge about the true mean. Therefore,
when using a very similar favorable event in the proof of their upper
bound, Mukherjee et al. (2017) have to bound the probability that an an
arm is not removed from the active set after the a certain number of
rounds -- even though the arm is above (respectively below) the
threshold. If this probability is low, then this means that the
algorithm is able to be certain about an arm's true mean \emph{quickly}
by employing variance estimates. In comparison, Locatelli et al (2016)
used bounds on on how often arms are drawn (\(T_i\)) to give a bound on
the probability of the favorable event. In order to derive the bound,
Mukherjee et al. (2017) use the Bernstein inequality.

It has to be noted that Mukherjee et al. (2017) have to assume that the
distributions have rewards bounded in \([0,1]\) in order to derive the
bound on the probability that an arm has not been eliminated correctly
in round \(T_i\). Thus, the distributions for which their upper bound
holds are a subset of the sub-Gaussian distributions considered in
Locatelli et al. (2016). However, it is not that the authors need to
assume that the rewards are in \([0,1]\), but this was assumed so that
the variance of the distribution is in \([0,1]\). Thus, the upper bound
holds in particular for Gaussian distributions with variance
\(\sigma^2 \in [0,1]\).

Based on this assumption, the authors show that the probability that an
arm has not been eliminated after \(m_i\) rounds is given by (compare
equation (2) in Mukherjee et al., 2017):

\begin{equation} \label{Mukherjee2017Equation2}
\mathbb{P}(\hat{\mu}_i > \mu_i + 2s_i) \leq \mathbb{P} (\hat{\mu}_i > \mu_i + 2\bar{s}_i) + \mathbb{P}(\hat{v}_i \geq \sigma^2_i + \sqrt{\rho \epsilon_{T_i}})
\end{equation}

Compare Mukherjee et al. (2017) for the notation. Then it becomes clear
that the left part of the RHS gives the probability that the mean
estimate is out of wack, while the right term gives the probability that
the variance is overestimated. The first term is bounded by the
Bernstein inequality. In the cumulative regret setting, there exists a
similar decomposition that bounds the expected number of pulls of a
suboptimal arm. Compare for example chapter 3.1 in ``Thompson Sampling:
An Asymptotically Optimal Finite Time Analysis'' by Kaufmann, Korda and
Munos (2012). There they decompose the expected number of draws of an
suboptimal arm for an optimistic (UCB like) into the probability that
the optimal arm's mean is underestimated, and the probability that the
UCB of the suboptimal arm is still very large. To give bounds for UCB
algorithms, it is then shown that these two terms show a certain
convergence. Mukherjee et al. do similarly for their variance based
algorithm in the thresholding setting.

(since the AugUCB eliminates arms, it is interesting to see how it
behaves when the underlying CTR changes over time\ldots{})

\subsection{New Approaches}\label{new-approaches}


\end{document}
